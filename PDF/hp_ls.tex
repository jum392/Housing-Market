\documentclass[10pt]{article}

% Document format
\usepackage{afterpage}
\usepackage[page]{appendix}
\usepackage{enumitem}
\usepackage{fullpage}
\usepackage{multirow}
\usepackage{natbib}
    \bibliographystyle{chicago}
\usepackage{pdflscape}
\usepackage{setspace}
    \doublespacing
    %\onehalfspacing
\usepackage{hyperref}
\usepackage{url}
\usepackage{verbatim}

% Graphics
\usepackage[dvipsnames,table]{xcolor}
    \definecolor{red1}{RGB}{242,220,219}
    \definecolor{red2}{RGB}{230,184,183}
    \definecolor{red3}{RGB}{218,150,148}
    \definecolor{red4}{RGB}{192, 80, 77}
    \definecolor{red5}{RGB}{150, 54, 52}
    \definecolor{red6}{RGB}{ 99, 37, 35}
\usepackage[capposition=top]{floatrow}
\usepackage{arydshln}
    \setlength\dashlinedash{0.5pt}
    \setlength\dashlinegap{1.5pt}
\usepackage{booktabs}
\usepackage{graphicx}
\usepackage{psfrag}
\usepackage[subrefformat=parens,labelformat=parens]{subcaption}
\usepackage{tikz}

% Mathematics
\usepackage{amsfonts}
\usepackage{amsmath}
\usepackage{amssymb}
\usepackage{amsthm}
    \newtheorem{proposition}{Proposition}
    \theoremstyle{definition}
    \newtheorem{definition}{Definition}
    \theoremstyle{remark}
    \newtheorem{remark}{Remark}
\usepackage{array}
\usepackage{bbm}
\usepackage{mathtools}

\title{House Prices and Labor Supply: Evidence from South Korea\thanks{[To be added]}}
\author{
    Inhyuk Choi\thanks{Email: \texttt{\href{mailto:ichoi@kipf.re.kr}{ichoi@kipf.re.kr}}.}
    \and
    Ji-Woong Moon\thanks{Email: \texttt{\href{mailto:jiwoong@mail.shufe.edu.cn}{jiwoong@mail.shufe.edu.cn}}.}
    }
\date{February 2020}

\begin{document}

\maketitle

\begin{abstract}
    [To be added] \\
    \textbf{JEL classification}: \\
    \textbf{Keywords}: 
\end{abstract}

\newpage

\section{Introduction}\label{sec:intro}
In many countries, including South Korea and the U.S., housing is the single most important asset for households. As such, it has been a primary interest for economists to explore how the housing market influences labor supply decisions. In this strand of literature, most papers are interested in estimating direct effects through wealth or collateral constraints on home-owners' balance sheets. However, housing market shocks influence labor supply decisions other than direct balance sheet effects. Specifically, unanticipated house price shocks affect the future housing costs, altering labor supply decisions through lifetime wealth effects, although these effects do not appear in the current balance sheets of households. This channel is particularly important since its' effects are likely to bigger for young workers, whose labor market decisions are critical for accumulating human capital and wealth. Using KLIPS data (one sentence data explanation), this paper explores how housing market shocks affect labor supply decisions of both home-owners and renters beyond their direct effects on balance sheets. In doing so, we focus on both extensive and intensive margins of labor supply and examine heterogeneous effects across ages and sectors. 

There are two challenges for identifying wealth effects through the expectation channel. First, identifying shocks from anticipated trends is difficult. Second, even if shocks are identified, it is not easy to estimate wealth effects working through the expectation channel. This is because housing market shocks can be correlated to other economic shocks and affect labor decisions through various channels. To deal with the first challenge, we construct the anticipated trend and unanticipated shock components from local house prices by regressing the price on its lagged variable and time, location fixed effects. In the main regression specification, we incorporate both trend and shock, and find that labor supply responds differently to the trend and shock. For the second challenge, we use the local house price instead of own house price to separate out the direct balance sheet effect. To control the local spending channel \citep{ms2014}, we estimate the labor supply responses of service and manufacturing sectors separately. For further robustness, we focus on a specific metropolitan region (i.e., Seoul) instead of estimating the effect for the whole country.



We contribute to the previous literature in the following ways. (In my opinion)
\begin{itemize}
	\item For Korean readers, we document how Korean labor market respond to housing market shocks.
	\item For general readers, we provide additional evidence on general patterns using Korean data. (...)
	\item (Haven't done yet) We identify(!) the effects of housing market shocks for young renters, which haven't been documented well (I guess). It is difficult to identify the effects for young renters because i) wealth effects for young renters depend on future house price expectation and moving is endogenous ii) housing market shocks are correlated to other economic shocks. How do we defend?
	\begin{itemize}
		\item For i), (Hypothetical argument. Need to be verified) In Korea, majority of population live in Seoul metropolitan area, and more so for young population. Thus, local housing shocks today are correlated more with future housing price in other large countries like the U.S ?  
		\item For ii) We only look at the manufacturing workers. (start career in manufacturing sectors) The rationale behind is that local economic shocks that affect both labor market and house prices do not directly affect manufacturing sectors employment because manufacturing firms produce goods sold nationally. We also look at Seoul sample, because as a one-day life zone, even services are (believed to be) tradable within Seoul. 
	\end{itemize}
\end{itemize}

\paragraph{Related literature.} There are many papers that study how housing ownership itself affects the labor supply, especially focusing on extensive margin of the labor supply (cf, Oswald hypothesis). I will refer the literature review of Broulikova et al. (2020).

\begin{quote}
    \textit{The empirical evidence concerning the effects of homeownership on unemployment is even more ambiguous (see Havet and Penot, 2010, for a review). Aggregate-level studies generally find a positive correlation between unemployment and the share of owner-occupied housing, both within and across countries (Blanchflower and Oswald, 2013; Green and Hendershott, 2001; Isebaert et al., 2015; Oswald, 1996). Individual-level studies, by contrast, tend to find that homeowners, if anything, do better on the job market than renters in terms of unemployment risk, its duration, and wages (Barceló, 2006; Battu et al., 2008; Coulson and Fisher, 2002; 2009; Flatau et al., 2003; Munch et al., 2006; 2008; Rouwendal and Nijkamp, 2010; Van Leuvensteijn and Koning, 2004).}
\end{quote}

There are many other paper that study housing wealth effects on labor supply. Zhao and Burge (2017) investigates the house wealth effects of elderly on labor supply, while using renters as control group.

\cite{DG2018} use local house price. It does exactly the same regression as ours. The primary objective of this paper is to identify housing wealth effect by using local house price and use renters as controlled group. They mention lots of considerations that help for us, and don't find significant effects for renters. We need to either modify the model, or find a reason why theirs and ours are different. They use real local house prices directly, while we use the residual shocks. Also, they do not control for the financial wealth of households. 

Campbell and Cocco (2007) studies housing wealth effects on consumption, and finds that the largest effects for old owners while the smallest effects for young renters. They use pseudo panel data, and do not include the financial wealth in the explanatory variables. Also, they use the real house price, rather than shocks. But, they have a structural model that endogenizes the ownership decision. Cunningham and Reed (2014) studies wage effects for high- vs low- levered households. He and Maire (2020) studies the liquidity effects of housing wealth on labor supply using a policy shock, and compare how high- vs low- liquidity household react differently.

\section{Data}\label{sec:data}
I use the KLIPS 04 - 22, since 04 is the first survey that monthly income is available. I consider the household head whose age is between 18 - 40. When calculating total wealth, I added housing deposits to financial wealth. It is a bit ambiguous that whether renters would report housing deposits as their wealth or not, because there is no housing deposit category (but, there exists personally rented money category) for wealth reporting. For house owners, if they rented a house, they should have reported the housing deposit as financial debts.

Table \ref{tab:sum_stat} presents summary statistics for the key variables used in the analysis.

\begin{table}[pt]
    \centering
    \caption{Summary Statistics}
    \label{tab:sum_stat}
    \begin{tabular}{llrrr}
        \toprule
        &  & Renters & Owners & All \\
        \midrule
        \multicolumn{4}{l}{\textbf{A. Wealth}} \\
        \multicolumn{2}{l}{\textit{Total wealth}} \\
        & (mean, unit?) & 1.22 & 3.65 & 1.95 \\
        \multicolumn{2}{l}{\textit{Financial debt}} \\
        & (mean, unit?) & 0.32 & 0.55 & 0.46 \\
        \phantom{} \\
        \multicolumn{4}{l}{\textbf{B. Employment}} \\
        \multicolumn{2}{l}{\textit{Employment status}} \\
        & 1 & 30.5 & 38.1 & 68.6 \\
        & 2 & 7.9 & 17.4 & 25.3 \\
        & 3 & 1.3 & 4.9 & 6.1 \\
        \multicolumn{2}{l}{\textit{Spouse's employment status}} \\
        & 1 &  \\
        & 2 &  \\
        & 3 &  \\
        \multicolumn{2}{l}{\textit{Hours of work}} \\
        & (mean, hours) & 43.9 & 42.6 & 43.2 \\
        \phantom{} \\
        \multicolumn{4}{l}{\textbf{C. Demographics}} \\
        \multicolumn{2}{l}{\textit{Age}} \\
        & less than 40 & 16.6 & 13.3 & 29.9 \\
        & 40 or above & 21.6 & 48.5 & 70.1 \\
        \multicolumn{2}{l}{\textit{Having children in school years}} \\
        & yes & 18.2 & 25.6 & 43.8 \\
        & no & 20.1 & 36.1 & 56.2 \\
        \midrule
        \multicolumn{2}{l}{$N$} & 53,011 & 85,556 & 138,567 \\
        \bottomrule
    \end{tabular}
    \floatfoot{\textit{Notes:} }
\end{table}

\section{Econometric Model}\label{sec:model}
It is necessary to separate out unanticipated-shocks and anticipated-trends from housing price dynamics to identify the effects of housing market shocks. In doing so, we do the following regression.
\begin{equation}
    \tilde P_{j,t} = \alpha_j + \gamma_t + \rho \tilde P_{j,t-1} + \epsilon_t
\end{equation}where $j$ is region, $\tilde P_{j,t}$ is the log-average house price calculated by KLIPS. $\alpha_j$ is the region-specific growth rate of housing price. Aggregate shock is absorbed by time-fixed effect $\gamma_t$. We run the fixed effect regression, and define the housing market shock $P_{j,t}$ by the residual: $P_{j,t} \equiv \tilde P_{j,t} - \hat{\tilde P}_{j,t}$.

\begin{equation}
    y_{it} = \alpha_i + \gamma_t + x_{it}'\beta + \eta_{j(i,t)} +\beta_0 I(H_{it} = r) + \sum_{h=o,r} P_{j(i,t)}\cdot I(H_{it} = h)\cdot\beta_{h} + \epsilon_{it}
\end{equation}
\begin{itemize}
	\item $y_{it}$: dependent variable. labor hours, unemployment dummy and real wages.
	\item $\alpha_i,\gamma_t$: individual and time fixed effects
	\item $x_{it}$: age, age-squared, total wealth (financial + housing), financial debt, monthly income (from all sources. labor, financial, etc.)
	\item $j(i,t)$: a region where $i$ lives at time $t$
	\item $\eta_{j(i,t)}$ : region fixed effect
	\item $P_{j(i,t)}$ : regional (real) house price excluding own house price
	\item $H_{it}$ : house ownership status. $H_{it} = o(wner)$ or $H_{it}=r(enter)$.
	\item $\beta_0$: coefficient for renter dummy
	\item $\beta_o,\beta_r$: effect of regional house prices on $y$, depending on house ownership status.
\end{itemize}

\section{Results}\label{sec:result}
\paragraph{Labor market participation} Model (1) is without interaction between regional house price and house ownership. The effect of home-ownership itself does not have a significant effect for labor hours.

Model (2) is baseline. The interaction term between renter and regional house price is positive and significant, while that of owners is negative and insignificant. The overall insignificant effect of model (1) comes from the higher labor hours of renters when regional house price is higher and lower labor hours of renters when regional house price is lower. Model (3) includes net wealth, rather than wealth and debt separate. Model (4) is only for people living in Seoul. The patterns are similar. The signs of wealth and debt effects are expected. 

\paragraph{Employment} (necessary?)

\paragraph{Hours of work} 

\begin{table}[pt]
    \centering
    \caption{Hours of work}
    \label{tab:hours_of_work}
    \begin{tabular}{lrrlrr}
        \toprule
        & \multicolumn{2}{r}{Young} &  & \multicolumn{2}{r}{Old} \\
        \cmidrule{2-3} \cmidrule{5-6} 
        & (1) & (2) &  & (3) & (4) \\
        & Children & No children &  & Children & No children \\
        \midrule
        reg HP cyc: renter & 0.354\phantom{***} & 0.544*** &  & $-$0.217\phantom{***} & $-$0.024\phantom{***} \\
        reg HP cyc: owner & 0.266\phantom{***} & 0.592\phantom{***} &  & $-$0.812\phantom{***} & $-$1.523*** \\
        tot wealth & $-$0.154\phantom{***} & 0.054\phantom{***} &  & 0.120\phantom{***} & $-$0.059\phantom{***} \\
        fin debt & 0.534*** & 0.572\phantom{***} &  & $-$0.200\phantom{***} & 0.091\phantom{***} \\
        \midrule
        $N$ & 3,738\phantom{***} & 4,391\phantom{***} &  & 4,931\phantom{***} & 8,541\phantom{***} \\
        \bottomrule
    \end{tabular}
    \floatfoot{\textit{Notes:} }
\end{table}

\paragraph{Wages} 

\paragraph{Consumption} 

\subsection{Robustness}\label{subsec:robust}


\section{Concluding Remarks}\label{sec:conc}


\newpage
\bibliography{hp_ls}

\newpage
\begin{appendices}
\numberwithin{equation}{section}

\section{Tables and Figures}


\end{appendices}

\end{document}